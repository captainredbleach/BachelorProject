\documentclass[a4paper,11pt,fleqn,oneside,openany]{memoir} 	% Openright aabner kapitler paa hoejresider (openany = vilkaarlig/begge)
%,draft

\usepackage{placeins}
%%%% PAKKER %%%%
% ¤¤ Oversaettelse og tegnsaetning ¤¤ %
\usepackage[utf8]{inputenc}					% Input-indkodning af tegnsaet, dvs. input fra keyboard, tegnoversigt eller andet (UTF8 = Unicode)
\usepackage[T1]{fontenc}					% Output-indkodning af tegnsaet, dvs. printede fonte og tegn (T1 = Type 1 font med support for de fleste europaeiske sprog)
\usepackage{lmodern}
\usepackage[english]{babel}					% Sproglig fremstilling af elementer (figur vs. figure, litteratur vs. bibliography osv.)
\usepackage{ragged2e,anyfontsize}			% Justering af elementer

\usepackage{multicol}

\usepackage{graphicx,wrapfig,lipsum}
\usepackage[newfloat]{minted}
\usepackage{xcolor}
% ¤¤ Figurer og tabeller (floats) ¤¤ %
\usepackage{graphicx} 						% Inkludering af eksterne billeder (JPG, PNG, PDF)
\usepackage{multirow}                		% Fletning af raekker og kolonner (\multicolumn og \multirow)
\usepackage{colortbl} 						% Farver i tabeller (fx \columncolor, \rowcolor og \cellcolor)

\usepackage{float}							% Muliggoer eksakt placering af floats, fx \begin{figure}[H]
\let\newfloat\relax 						% Justering mellem float-pakken og memoir
%\usepackage{eso-pic}						% Tilfoej billedekommandoer paa hver side
%\usepackage{wrapfig}						% Indsaettelse af figurer omsvoebt af tekst 
%\usepackage{multicol}         	        	% Muliggoer tekst i spalter
%\usepackage{rotating}						% Rotation af tekst med \begin{sideways}...\end{sideways}

% ¤¤ Matematik mm. ¤¤
\usepackage{amsmath,amssymb,stmaryrd} 		% Avancerede matematik-udvidelser
\usepackage{mathtools}						% Andre matematik- og tegnudvidelser
\usepackage{textcomp}                 		% Symbol-udvidelser (fx promille-tegn med \textperthousand)
\usepackage[binary-units]{siunitx}						% Flot og konsistent praesentation af tal og enheder med \si{enhed} og \SI{tal}{enhed}
\sisetup{output-decimal-marker = {,}}		% Opsaetning af \SI og decimalseparator
%\usepackage[version=3]{mhchem} 			% Kemi-pakke til flot og let notation af formler, fx \ce{Fe2O3}
%\usepackage{rsphrase}						% Kemi-pakke til RS-saetninger, fx \rsphrase{R1}

% ¤¤ Referencer og kilder ¤¤ %
\usepackage[danish]{varioref}				% Muliggoer bl.a. krydshenvisninger med sidetal (\vref)
\usepackage{natbib}							% Udvidelse med naturvidenskabelige citationsmodeller, herunder Harvard-modellen
%\usepackage{xr}							% Referencer til eksternt dokument med \externaldocument{<NAVN>}
%\usepackage[acronym, toc]{glossaries}					% Terminologi- eller symbolliste (se mere i Lars Madsens Latex-bog)
\usepackage[nomain, acronym, toc, nonumberlist]{glossaries-extra}
\makeglossaries
% \newacronym{gcd}{GCD}{Greatest Common Divisor}

% \newacronym{lcm}{LCM}{Least Common Multiple}

\newacronym{sdr}{SDR}{Software Defined Radio}
\newacronym{iso}{ISO}{International Organisation of Standardisation}


% \newglossaryentry{Software Defined Radio}
% {
%     name=Software Defined Radio,
%     description= {A software defined radio is based on an antenna, a receiver front end and the mixer software run on a processor. The main difference between typical (hardware) radios and Software defined radios is that the signal processing is performed in software instead of on hardware, which requires varying amounts of computing power depending on the sampling rate, as the system is digital instead of analogue. Some of the benefits in space applications include the ability to re-tune or re-purpose the radio through software.}
% }





%\setabbreviationstyle[acronym]{long-short}

% ¤¤ Misc. ¤¤ %

\usepackage{listings}						% Placer kildekode i dokumentet med \begin{lstlisting}...\end{lstlisting}
\usepackage{lipsum}							% Dummy tekst med fx \lipsum[2]
\usepackage[shortlabels]{enumitem}			% Muliggoer enkelt konfiguration af lister (se \setlist nedenfor)
\usepackage{pdfpages}						% Goer det muligt at inkludere pdf-dokumenter med kommandoen \includepdf[pages={x-y}]{fil.pdf}	
\pdfoptionpdfminorversion=6					% Muliggoer inkludering af pdf-dokumenter af version 1.6 og hoejere
\pretolerance=2500 							% Justering af afstand mellem ord (hoejt tal, mindre orddeling og mere luft mellem ord)

% Kommentarer og rettelser med \fxnote. Med 'final' i stedet for 'draft' udloeser hver note en error i den faerdige rapport.
\usepackage[footnote,draft,danish,silent,nomargin]{fixme}		


%%%% BRUGERDEFINEREDE INDSTILLINGER %%%%

% ¤¤ Marginer ¤¤ %
\setlrmarginsandblock{3.5cm}{2.5cm}{*}		% \setlrmarginsandblock{Indbinding}{Kant}{Ratio}
\setulmarginsandblock{2.5cm}{3.0cm}{*}		% \setulmarginsandblock{Top}{Bund}{Ratio}
\checkandfixthelayout 						% Oversaetter vaerdier til brug for andre pakker

%	¤¤ Afsnitsformatering ¤¤ %
\setlength{\parindent}{0mm}           		% Stoerrelse af indryk
\setlength{\parskip}{3mm}          			% Afstand mellem afsnit ved brug af double Enter
\linespread{1,1}							% Linjeafstand

% ¤¤ Litteraturlisten ¤¤ %
\bibpunct[,]{[}{]}{;}{a}{,}{,} 				% Definerer parametre ved Harvard-henvisning (bl.a. parantestype og seperatortegn)
\bibliographystyle{harvard}			% Udseende af litteraturlisten (Harvard-metoden - skift til fx 'plain' for tal)

% ¤¤ Dybde af overskrifter ¤¤ %
\setsecnumdepth{subsection}		 			% Dybden af nummerede overkrifter (part/chapter/section/subsection)
\settocdepth{subsection} 					% Dybden af overskrifter vist i indholdsfortegnelsen

% ¤¤ Lister ¤¤ %
\setlist{
  topsep=0pt,								% Vertikal afstand mellem tekst og listen
  itemsep=-1ex,								% Vertikal afstand mellem items
} 

% ¤¤ Visuelle referencer ¤¤ %
%\usepackage[colorlinks]{hyperref}			% Danner klikbare referencer (hyperlinks) i dokumentet
\PassOptionsToPackage{hyphens}{url}\usepackage[colorlinks]{hyperref}
\hypersetup{colorlinks = true,				% Opsaetning af farvede hyperlinks (interne links, citeringer og URL)
    linkcolor = black,
    citecolor = black,
    urlcolor = black
}

% ¤¤ Opsaetning af figur- og tabeltekst ¤¤ %
\captionnamefont{\small\bfseries\itshape}	% Opsaetning af tekstdelen ('Figur' eller 'Tabel')
\captiontitlefont{\small}					% Opsaetning af nummerering
\captiondelim{. }							% Seperator mellem nummerering og figurtekst
\captionstyle{\centering}					% Justering/placering af figurteksten (centreret = \centering, venstrejusteret = \raggedright)
\captionwidth{\linewidth}					% Bredden af figurteksten
\hangcaption								% Venstrejusterer fler-linjers figurtekst under hinanden
\setlength{\belowcaptionskip}{0pt}			% Afstand under figurteksten
\usepackage{enumitem}
\newcommand{\subscript}[2]{$#1 _ #2$}
		
% ¤¤ Opsaetning af listings ¤¤ %
\definecolor{commentGreen}{RGB}{34,139,24}
\definecolor{stringPurple}{RGB}{208,76,239}

\lstset{language=Matlab,					% Sprog
	basicstyle=\ttfamily\scriptsize,		% Opsaetning af teksten
	keywords={for,if,while,else,elseif,		% Noegleord at fremhaeve
			  end,break,return,case,
			  switch,function},
	keywordstyle=\color{blue},				% Opsaetning af noegleord
	commentstyle=\color{commentGreen},		% Opsaetning af kommentarer
	stringstyle=\color{stringPurple},		% Opsaetning af strenge
	showstringspaces=false,					% Mellemrum i strenge enten vist eller blanke
	numbers=left, numberstyle=\tiny,		% Linjenumre
	extendedchars=true, 					% Tillader specielle karakterer
	columns=flexible,						% Kolonnejustering
	breaklines, breakatwhitespace=true,		% Bryd lange linjer
}

% ¤¤ Navngivning ¤¤ %
\addto\captionsdanish{
	\renewcommand\contentsname{Contents}			% Skriver 'Indholdsfortegnelse' i stedet for 'Indhold'
	\renewcommand\appendixname{Appendix}					% Skriver 'Appendiks' i stedet for 'Appendix'
	\renewcommand\appendixpagename{Appendix}
	\renewcommand\appendixtocname{Appendix}
					% Skriver 'Kapitel' foran kapitlerne i indholdsfortegnelsen
				% Skriver 'Appendiks' foran appendiks i indholdsfortegnelsen
}

% ¤¤ Kapiteludssende ¤¤ %
\definecolor{numbercolor}{gray}{0.7}		% Definerer en farve til brug til kapiteludseende
\newif\ifchapternonum

\makechapterstyle{jenor}{					% Definerer kapiteludseende frem til ...
  \renewcommand\beforechapskip{0pt}
  \renewcommand\printchaptername{}
  \renewcommand\printchapternum{}
  \renewcommand\printchapternonum{\chapternonumtrue}
  \renewcommand\chaptitlefont{\fontfamily{pbk}\fontseries{db}\fontshape{n}\fontsize{25}{35}\selectfont\raggedleft}
  \renewcommand\chapnumfont{\fontfamily{pbk}\fontseries{m}\fontshape{n}\fontsize{1in}{0in}\selectfont\color{numbercolor}}
  \renewcommand\printchaptertitle[1]{%
    \noindent
    \ifchapternonum
    \begin{tabularx}{\textwidth}{X}
    {\let\\\newline\chaptitlefont ##1\par} 
    \end{tabularx}
    \par\vskip-2.5mm\hrule
    \else
    \begin{tabularx}{\textwidth}{Xl}
    {\parbox[b]{\linewidth}{\chaptitlefont ##1}} & \raisebox{-15pt}{\chapnumfont \thechapter}
    \end{tabularx}
    \par\vskip2mm\hrule
    \fi
  }
}											% ... her

\chapterstyle{jenor}						% Valg af kapiteludseende - Google 'memoir chapter styles' for alternativer

% ¤¤ Sidehoved/sidefod ¤¤ %

\makepagestyle{Uni}							% Definerer sidehoved og sidefod udseende frem til ...
\makepsmarks{Uni}{%
	\createmark{chapter}{left}{shownumber}{}{. \ }
	\createmark{section}{right}{shownumber}{}{. \ }
	\createplainmark{toc}{both}{\contentsname}
	\createplainmark{lof}{both}{\listfigurename}
	\createplainmark{lot}{both}{\listtablename}
	\createplainmark{bib}{both}{\bibname}
	\createplainmark{index}{both}{\indexname}
	\createplainmark{glossary}{both}{\glossaryname}
}
\nouppercaseheads											% Ingen Caps oenskes

\makeevenhead{Uni}{Group B217}{}{\leftmark}				% Lige siders sidehoved (\makeevenhead{Navn}{Venstre}{Center}{Hoejre})
\makeoddhead{Uni}{\rightmark}{}{Aalborg University}			% Ulige siders sidehoved (\makeoddhead{Navn}{Venstre}{Center}{Hoejre})
\makeevenfoot{Uni}{\thepage}{}{}							% Lige siders sidefod (\makeevenfoot{Navn}{Venstre}{Center}{Hoejre})
\makeoddfoot{Uni}{}{}{\thepage}								% Ulige siders sidefod (\makeoddfoot{Navn}{Venstre}{Center}{Hoejre})
\makeheadrule{Uni}{\textwidth}{0.5pt}						% Tilfoejer en streg under sidehovedets indhold
\makefootrule{Uni}{\textwidth}{0.5pt}{1mm}					% Tilfoejer en streg under sidefodens indhold

\copypagestyle{Unichap}{Uni}								% Der dannes en ny style til kapitelsider
\makeoddhead{Unichap}{}{}{}									% Sidehoved defineres som blank på kapitelsider
\makeevenhead{Unichap}{}{}{}
\makeheadrule{Unichap}{\textwidth}{0pt}
\aliaspagestyle{chapter}{Unichap}							% Den ny style vaelges til at gaelde for chapters
															% ... her
															
\pagestyle{Uni}												% Valg af sidehoved og sidefod (benyt 'plain' for ingen sidehoved/fod)


%%%% EGNE KOMMANDOER %%%%

% ¤¤ Billede hack ¤¤ %										% Indsaet figurer nemt med \figur{Stoerrelse}{Fil}{Figurtekst}{Label}
\newcommand{\figur}[4]{
		\begin{figure}[H] \centering
			\includegraphics[width=#1\textwidth]{billeder/#2}
			\caption{#3}
			\label{#4}
		\end{figure} 
}
% skriv \n fx
\newcommand*{\escape}[1]{\texttt{\textbackslash #1}}

% Commands for making Use Cases
\newcommand\tabularhead[1]{
\begin{table}[H]
    \caption{<<#1>>}  
    \begin{tabular}{|p{0.18\textwidth}|p{0.75\textwidth}|}
    \hline
    \textbf{#1} & \textbf{} \\
    \hline
    }

  \newcommand\addrow[2]{#1 &#2\\ \hline}

  \newcommand\addmulrow[2]{\begin{minipage}[t][][t]{2.5cm}#1\end{minipage}% 
     &\begin{minipage}[t][][t]{8cm}
      \begin{enumerate} #2\end{enumerate}
      \end{minipage}\\}

\newenvironment{usecase}{\tabularhead}
{\end{tabular} \end{table}}
%\hline
% End of Commands


% ¤¤ Specielle tegn ¤¤ %
\newcommand{\dec}{^{\circ}}									% '\dec' returnerer et gradtegn (husk $$ udenfor aligns)
\newcommand{\decC}{^{\circ}\text{C}}						% '\decC' returnerer et gradtegn + 'C' (husk $$ udenfor aligns)
\newcommand{\m}{\cdot}										% '\m' returnerer et gangetegn


%%%% ORDDELING %%%%

\hyphenation{In-te-res-se e-le-ment}

\definecolor{PastelGreen}{HTML}{D5E8D4}
\definecolor{PastelBlue}{HTML}{DAE8FC}
\definecolor{PastelRed}{HTML}{F8CECC}
\definecolor{PastelOrange}{HTML}{FFE6CC}
\definecolor{PastelYellow}{HTML}{FFF2CC}
\definecolor{PastelPurple}{HTML}{E1D5E7}
\definecolor{PastelGrey}{HTML}{E5E5E5}
\definecolor{PastelDarkGrey}{HTML}{C5C5C5}

%\usepackage[disable]{todonotes}
\usepackage[colorinlistoftodos]{todonotes}



\usepackage{caption}
\usepackage{subcaption}
\usepackage{plantuml}

\usepackage{svg}
\usepackage{amsmath}
\usepackage{commath}